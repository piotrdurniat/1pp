\documentclass[a4paper,12pt]{article}
\usepackage[left=2cm,right=2cm,top=2cm,bottom=2cm]{geometry} % Do ustawień marginesów
\usepackage{multicol} % Dla podziału na kolumny
\usepackage{ragged2e} % Dla justowania tekstu
\usepackage{graphicx} % Required for inserting images
\usepackage{float}
\usepackage{caption}
\usepackage{amsmath} % Math formulas
\usepackage{amssymb} % Symbols
\usepackage[svgnames]{xcolor}
\usepackage[colorlinks=true, urlcolor=blue, linkcolor=black, citecolor=orange]{hyperref} % Hyperlinks
\usepackage{polski} % Polish language
\usepackage[utf8]{inputenc} % Text encoding
\usepackage{enumitem} % Pakiet do elastycznego sterowania listami
\usepackage{indentfirst}
\usepackage{array}

\begin{document}

% Górna część strony
\noindent
\begin{minipage}{0.5\textwidth}
    \raggedright
    \textbf{Piotr Durniat} \\
    I rok, Fizyka \\
    Wtorek, 8:00-10:15 \\
    \vspace{0.5cm}
    \vspace{0.5cm}
\end{minipage}%
\begin{minipage}{0.5\textwidth}
    \raggedleft
    Data wykonania pomiarów: \\
    18.03.2025 \\
    \vspace{0.5cm}
    Prowadząca: \\
    dr Iwona Mróz
\end{minipage}

% ------------------------------------------------------------

% Tytuł ćwiczenia
\vspace{2cm}
\begin{center}
    \LARGE \textbf{Ćwiczenie nr 11} \\[0.5cm]
    \Large \textbf{Wyznaczanie elipsoidy bezwładności bryły sztywnej}
\end{center}

\vspace{1cm}
\noindent

\tableofcontents
\newpage


% ------------------------------------------------------------

% --------- WSTĘP TEORETYCZNY ---------
\section{Wstęp teoretyczny}

Wahadło torsyjne wychylone o kąt $\theta$ od położenia równowagi wykonuje drgania skrętne pod wpływem momentu siły sprężystości $M$:

\begin{equation}
    M = -k\theta
\end{equation}


Okres drgań wahadła torsyjnego:
\begin{equation} \label{eq:okres_drgan_w_zaleznosci_od_momentu_bezwladnosci}
    T = 2\pi \sqrt{\frac{I}{k}}
\end{equation}

gdzie:
\begin{itemize}
    \item $I$ - moment bezwładności bryły
    \item $k$ - moment kierujący wahadła
\end{itemize}

Stąd:
\begin{equation} \label{eq:moment_bezwladnosci_w_zaleznosci_od_okresu}
    I = k \left(\frac{T}{2\pi}\right)^2
\end{equation}

Dryński - Rozdział 19
% --------- OPIS DOŚWIADCZENIA ---------
\section{Opis doświadczenia}




\section{Opracowanie wyników pomiarów}

% ---------- TABELE ----------
\subsection{Tabele pomiarowe}

\begin{itemize}
    \item Błąd wskazania zerowego suwmiarki wyniósł 0.15 mm.
    \item Niepewność wzorcowania suwmiarki $\Delta_d D = 0.05$ mm.
    \item Kąt $\alpha = 30^\circ$.
\end{itemize}

\begin{table}[H]
    \centering
    \begin{tabular}{|l|c|c|}
        \hline
        \textbf{Rodzaj układu} & \textbf{Czas 10 drgań [s]} & \textbf{Okres T [s]} \\
        \hline
        Sama ramka & 18.135 & 1.8135 \\
        \hline
        Ramka + walec wzorcowy & 25.192 & 2.5192 \\
        \hline
    \end{tabular}
    \caption{Pomiar czasu drgań ramki i ramki z walcem wzorcowym}
    \label{tab:pomiar_czasu_drgan_ramki_z_walcem}
\end{table}

\begin{table}[H]
    \centering
    \begin{tabular}{|l|c|c|}
        \hline
        \textbf{Oś obrotu} & \textbf{Czas 10 drgań [s]} & \textbf{Okres T [s]} \\
        \hline
        Główna oś 1 (x) & 23.001 & 2.3001 \\
        \hline
        Główna oś 2 (y) & 23.008 & 2.3008 \\
        \hline
        Główna oś 3 (z) & 22.937 & 2.2937 \\
        \hline
        Dowolna oś przez środek masy (s) & 23.006 & 2.3006 \\
        \hline
        Dowolna oś nieprzechodząca przez środek masy (t) & 22.573 & 2.2573 \\
        \hline
    \end{tabular}
    \caption{Pomiar czasu drgań ramki z badaną bryłą dla różnych osi}
    \label{tab:pomiar_czasu_drgan_badanej_bryly}
\end{table}

\begin{table}[H]
    \centering
    \begin{tabular}{|l|c|c|}
        \hline
        \textbf{Wielkość} & \textbf{Wartość [mm]} & \textbf{Po korekcie [mm]} \\
        \hline
        Średnica podstawy $d$ [mm] & 60.15 & 60.00 \\
        \hline
        Wysokość walca $h$ [mm] & 60.15 & 60.00\\
        \hline
        Masa walca $m$ [g] & 1330 & 1330 \\
        \hline
    \end{tabular}
    \caption{Rozmiary bryły wzorcowej, wraz z korektą wskazania zerowego}
    \label{tab:rozmiar_bryly_wzorcowej}
\end{table}

% ---------- OBLICZENIA ----------
\subsection{Moment bezwładności bryły wzorcowej}

Badaną bryłą jest walec o wymiarach w tabeli \ref{tab:rozmiar_bryly_wzorcowej}, moment bezwładności walca dla osi przechodzącej przez środki podstaw określa wzór:

\begin{equation} \label{eq:moment_bezwladnosci_walca}
    I_w = \frac{1}{2}mr^2 = \frac{1}{2}m\left(\frac{d}{2}\right)^2
\end{equation}

Podstawiając dane z tabeli \ref{tab:rozmiar_bryly_wzorcowej} otrzymujemy:

\begin{equation} \label{eq:moment_bezwladnosci_walca_wartosc}
    I_w = \frac{1}{2} \cdot 1,330 \cdot \left(\frac{0.06}{2}\right)^2 = 0.0060\,\text{kgm}^2
\end{equation}


\subsection{Moment bezwładności badanej bryły}

Dla samej ramki okres drgań na podstawie wzoru \ref{eq:okres_drgan_w_zaleznosci_od_momentu_bezwladnosci} wynosi:

\begin{equation} \label{eq:okres_drgan_ramki}
    T_0 = 2\pi \sqrt{\frac{I_0}{k}}
\end{equation}

Stąd okres $T_w$ drgań ramki z bryłą wzorcową o momencie bezwładności $I_w$ wynosi:

\begin{equation} \label{eq:okres_drgan_ramki_z_walcem}
    T_w = 2\pi \sqrt{\frac{I_0 + I_w}{k}}
\end{equation}

Zatem z równań \eqref{eq:okres_drgan_ramki} i \eqref{eq:okres_drgan_ramki_z_walcem} moment kierujący wahadła wynosi:

\begin{equation} \label{eq:moment_kierujacy_w_zaleznosci_od_okresu}
    k = \frac{4\pi^2 I_w}{T_w^2 - T_0^2}
\end{equation}

Z wzorów \eqref{eq:moment_bezwladnosci_w_zaleznosci_od_okresu} i \eqref{eq:moment_kierujacy_w_zaleznosci_od_okresu} otrzymujemy moment bezwładności $I_x$ badanej bryły:

\begin{equation} \label{eq:moment_bezwladnosci_badanej_bryly}
    I_x = \frac{T_x^2 - T_0^2}{T_w^2 - T_0^2} \cdot I_w
\end{equation}

Po podstawieniu okresów drgań z tabel \ref{tab:pomiar_czasu_drgan_badanej_bryly} oraz \ref{tab:pomiar_czasu_drgan_ramki_z_walcem} i momentu bezwładności walca $I_w$ z wzoru \eqref{eq:moment_bezwladnosci_walca_wartosc} otrzymujemy:

\begin{table}[H]
    \centering
    \begin{tabular}{|c|c|}
        \hline
        Oś & Moment bezwładności [$\text{kg}\cdot\text{m}^2$] \\
        \hline
        $I_x$ & 0,000392 \\
        $I_y$ & 0,000392 \\
        $I_z$ & 0,000386 \\
        $I_s$ & 0,000392 \\
        $I_t$ & 0,000354 \\
        \hline
    \end{tabular}
    \caption{Momenty bezwładności względem poszczególnych osi}
    \label{tab:momenty_bezwladnosci}
\end{table}

Przykładowe obliczenia dla $I_x$:

\begin{equation*}
    I_x = \frac{(2,3001)^2 - (1,8135)^2}{(2,5192)^2 - (1,8135)^2} \cdot 0,005985 = 0,000392\,\text{kgm}^2
\end{equation*}


% 2. Obliczyć momenty bezwładności badanej bryły względem wszystkich osi obrotu, dla których mierzono T1.

\subsection{Wyznaczanie elipsoidy bezwładności badanej bryły}


Równanie elipsoidy ma postać:

\begin{equation*}
    \frac{x^2}{a^2} + \frac{y^2}{b^2} + \frac{z^2}{c^2} = 1
\end{equation*}

a, b, c to półosie elipsoidy, które są zdefiniowane jako:

\begin{equation*}
    a = \frac{1}{\sqrt{I_x}}, \quad
    b = \frac{1}{\sqrt{I_y}}, \quad
    c = \frac{1}{\sqrt{I_z}}
\end{equation*}

Podstawiając momenty bezwładności z tabeli \ref{tab:momenty_bezwladnosci} otrzymujemy:

\begin{equation*}
    a=50,519636 \text{m} \quad
    b=50,479042 \text{m} \quad
    c=50,894785 \text{m}
\end{equation*}

\subsection{Wyznaczanie momentu bezwładności względem osi niebędących głównymi osiami bezwładności}

% 4. Wyznaczyć długość odcinka Ri łączącego początek układu współrzędnych z punktem przebicia P elipsoidy bezwładności przez wskazaną przez prowadzącego oś, nie będącą osią główną. Na tej podstawie obliczyć moment bezwładności bryły względem tej osi.

% My tu chyba mamy dla obydu osi to zrobić (nieprzechodząca przez środek masy i przechodząca przez środek masy)
\subsubsection{Oś przechodząca przez środek masy (s)}

Znając momenty bezwładności względem głównych osi możemy obliczyć momenty bezwładności względem osi niebędących głównymi osiami bezwładności, korzystając z wzoru:

\begin{equation*}
    I_i = \frac{1}{R_i^2}
\end{equation*}

gdzie $R_i$ to odległość między początkiem układu współrzędnych a punktem przebicia elipsoidy bezwładności przez wybraną oś.

Wybrana oś przechodzi przez dwa przeciwległe wierzchołki bryły, stąd współrzędne jednego z punktów leżącego na osi $s$ wynoszą:

\begin{equation*}
    x = ..., \quad
    y = ..., \quad
    z = ...
\end{equation*}




\subsection{Porównanie momentów bezwładności obliczonych dwoma sposobami}

% 5. Porównać momenty bezwładności (dla tej samej osi) obliczone wg. punktu 2 i punktu 4.


% ---------- NIEPEWNOŚCI ----------
\section{Ocena niepewności pomiaru}

\subsection{Niepewność standardowa momentu bezwładności}

% 6. Obliczyć niepewność standardową wyznaczenia momentu bezwładności według punktu 2

% ---------- WNIOSKI ----------
\section{Wnioski}

% 7. Przeanalizować jakie czynniki mają największy wpływ na jej [niepewności wyznaczania momentu bezwładności] wartość. Sformułować wnioski.

\section{Wykresy}

\bibliographystyle{plain}
\bibliography{bibliography}

\end{document}
