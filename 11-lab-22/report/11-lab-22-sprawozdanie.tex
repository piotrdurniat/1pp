\documentclass[a4paper,12pt]{article}
\usepackage[left=2cm,right=2cm,top=2cm,bottom=2cm]{geometry} % Do ustawień marginesów
\usepackage{multicol} % Dla podziału na kolumny
\usepackage{ragged2e} % Dla justowania tekstu
\usepackage{graphicx} % Required for inserting images
\usepackage{float}
\usepackage{caption}
\usepackage{amsmath} % Math formulas
\usepackage{amssymb} % Symbols
\usepackage[svgnames]{xcolor}
\usepackage[colorlinks=true, urlcolor=blue, linkcolor=black, citecolor=orange]{hyperref} % Hyperlinks
\usepackage{polski} % Polish language
\usepackage[utf8]{inputenc} % Text encoding
\usepackage{enumitem} % Pakiet do elastycznego sterowania listami
\usepackage{indentfirst}
\usepackage{array}
\usepackage{longtable}
\setlist[itemize]{itemsep=0pt, topsep=0pt}

\begin{document}

% Górna część strony
\noindent
\begin{minipage}{0.5\textwidth}
    \raggedright
    \textbf{Piotr Durniat} \\
    I rok, Fizyka \\
    Wtorek, 8:00-10:15 \\
    \vspace{0.5cm}
    \vspace{0.5cm}
\end{minipage}%
\begin{minipage}{0.5\textwidth}
    \raggedleft
    Data wykonania pomiarów: \\
    21.05.2025 \\
    \vspace{0.5cm}
    Prowadząca: \\
    dr Iwona Mróz
\end{minipage}

% Tytuł ćwiczenia
\vspace{2cm}
\begin{center}
    \LARGE \textbf{Ćwiczenie nr 22} \\[0.5cm]
    \Large \textbf{Pomiar wilgotności powietrza atmosferycznego}
\end{center}

% Reszta treści
\vspace{1cm} % Kolejny odstęp
\noindent

\tableofcontents
\newpage

% ---------- WSTĘP TEORETYCZNY ----------
\section{Wstęp teoretyczny}

% ---------- OPIS DOŚWIADCZENIA ----------
\section{Opis doświadczenia}

% ---------- OPRACOWANIE WYNIKÓW POMIARÓW ----------
\section{Opracowanie wyników pomiarów}

% ---------- TABELE ----------
\subsection{Tabele pomiarowe}


\begin{table}[H]
    \centering
    \begin{tabular}{|c|c|}
        \hline
        \textbf{Nr. pomiaru} & \textbf{Wilgotność} \\
        \hline
        1 & 59\% \\
        \hline
        2 & 61\% \\
        \hline
    \end{tabular}
    \caption{Wilgotność powietrza odczytana z higrometru włosowego.}
\end{table}

\begin{table}[H]
    \centering
    \begin{tabular}{|c|c|c|c|}
        \hline
        $T_s\,[^\circ C]$ & $T_m\,[^\circ C]$ & $\Delta T\,[^\circ C]$ & wilgotność względna \\
        \hline
        25{,}00 & 20{,}00 & 5{,}00 & 63{,}00\% \\
        \hline
    \end{tabular}
    \caption{Wyniki pomiarów dla Psychrometru Assmanna.}
\end{table}

\begin{table}[H]
    \centering
    \begin{tabular}{|c|c|c|}
        \hline
        Nr. pomiaru & $T_1\,[^\circ C]$ & $T_2\,[^\circ C]$ \\
        \hline
        1 & 11{,}4 & 14{,}0 \\
        \hline
        2 & 12{,}0 & 14{,}7 \\
        \hline
        3 & 11{,}9 & 16{,}0 \\
        \hline
        4 & 12{,}3 & 13{,}8 \\
        \hline
        5 & 12{,}5 & 14{,}5 \\
        \hline
        6 & 12{,}5 & 14{,}1 \\
        \hline
        7 & 12{,}5 & 13{,}5 \\
        \hline
        8 & 11{,}9 & 14{,}4 \\
        \hline
        9 & 11{,}5 & 15{,}7 \\
        \hline
        10 & 12{,}0 & 14{,}9 \\
        \hline
        11 & 12{,}0 & 15{,}4 \\
        \hline
        12 & 12{,}3 & 14{,}9 \\
        \hline
        13 & 11{,}8 & 15{,}4 \\
        \hline
        14 & 12{,}4 & 15{,}1 \\
        \hline
        15 & 12{,}2 & 14{,}2 \\
        \hline
    \end{tabular}
    \caption{Wyniki pomiarów metodą punktu rosy.}
\end{table}


% ---------- OBLICZENIA ----------
\subsection{...}

% ---------- NIEPEWNOŚCI ----------
\section{Ocena niepewności pomiaru}

% ---------- WNIOSKI ----------
\section{Wnioski}

% ---------- WYKRESY ----------
\section{Wykresy}

\bibliographystyle{plain}
\bibliography{bibliography}

\end{document}
