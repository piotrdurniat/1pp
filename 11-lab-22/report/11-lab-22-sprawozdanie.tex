\documentclass[a4paper,12pt]{article}
\usepackage[left=2cm,right=2cm,top=2cm,bottom=2cm]{geometry} % Do ustawień marginesów
\usepackage{multicol} % Dla podziału na kolumny
\usepackage{ragged2e} % Dla justowania tekstu
\usepackage{graphicx} % Required for inserting images
\usepackage{float}
\usepackage{caption}
\usepackage{amsmath} % Math formulas
\usepackage{amssymb} % Symbols
\usepackage[svgnames]{xcolor}
\usepackage[colorlinks=true, urlcolor=blue, linkcolor=black, citecolor=orange]{hyperref} % Hyperlinks
\usepackage{polski} % Polish language
\usepackage[utf8]{inputenc} % Text encoding
\usepackage{enumitem} % Pakiet do elastycznego sterowania listami
\usepackage{indentfirst}
\usepackage{array}
\usepackage{longtable}
\setlist[itemize]{itemsep=0pt, topsep=0pt}

\begin{document}

% Górna część strony
\noindent
\begin{minipage}{0.5\textwidth}
    \raggedright
    \textbf{Piotr Durniat} \\
    I rok, Fizyka \\
    Wtorek, 8:00-10:15 \\
    \vspace{0.5cm}
    \vspace{0.5cm}
\end{minipage}%
\begin{minipage}{0.5\textwidth}
    \raggedleft
    Data wykonania pomiarów: \\
    21.05.2025 \\
    \vspace{0.5cm}
    Prowadząca: \\
    dr Iwona Mróz
\end{minipage}

% Tytuł ćwiczenia
\vspace{2cm}
\begin{center}
    \LARGE \textbf{Ćwiczenie nr 22} \\[0.5cm]
    \Large \textbf{Pomiar wilgotności powietrza atmosferycznego}
\end{center}

% Reszta treści
\vspace{1cm} % Kolejny odstęp
\noindent

\tableofcontents
\newpage

% ---------- WSTĘP TEORETYCZNY ----------
\section{Wstęp teoretyczny}

% ---------- OPIS DOŚWIADCZENIA ----------
\section{Opis doświadczenia}

Celem doświadczenia było wyznaczenie wilgotności względnej powietrza atmosferycznego za pomocą trzech różnych metod pomiarowych.

\subsection{Metoda punktu rosy}

W metodzie punktu rosy wykorzystano efekt Peltiera do obniżania temperatury płytki krzemowej. Powierzchnia płytki była oświetlana wiązką laserową. Pomiar polegał na:
\begin{itemize}
    \item Stopniowym zwiększaniu natężenia prądu zasilającego element Peltiera, co powodowało obniżanie temperatury płytki
    \item Obserwacji powierzchni płytki i odnotowaniu temperatury $T_1$, przy której pojawia się mgiełka (rozproszenie światła laserowego)
    \item Zmniejszaniu prądu i odnotowaniu temperatury $T_2$, przy której mgiełka znika
\end{itemize}

Wykonano 15 pomiarów, każdorazowo zmieniając natężenie prądu wokół punktu rosy. Temperatura punktu rosy została obliczona jako średnia z temperatur pojawienia się i zniknięcia mgiełki. Następnie, korzystając z tablic ciśnienia pary nasyconej, obliczono wilgotność względną ze wzoru:
\begin{equation}
    S = \frac{p_t}{p_0}
\end{equation}
gdzie $p_t$ to ciśnienie pary nasyconej przy temperaturze punktu rosy, a $p_0$ to ciśnienie pary nasyconej przy temperaturze otoczenia.

\subsection{Psychrometr Assmanna}

W pomiarach wykorzystano psychrometr Assmanna wyposażony w dwa termometry: suchy i mokry. Procedura pomiaru obejmowała:
\begin{itemize}
    \item Napełnienie próbówki wodą destylowaną
    \item Wsunięcie próbówki do osłony termometru oznaczonego kolorem niebieskim (termometr mokry)
    \item Nakręcenie mechanizmu dmuchawy
    \item Odczytanie wskazań obu termometrów po około 4 minutach
\end{itemize}

Na podstawie różnicy temperatur między termometrem suchym ($T_s$) i mokrym ($T_m$) obliczono wilgotność względną korzystając z tablic psychrometrycznych.

\subsection{Higrometr włosowy}

Pomiar wilgotności przy użyciu higrometru włosowego polegał na bezpośrednim odczycie wskazań przyrządu. Wykonano dwa odczyty: na początku i na końcu sesji pomiarowej.

% ---------- OPRACOWANIE WYNIKÓW POMIARÓW ----------
\section{Opracowanie wyników pomiarów}

% ---------- TABELE ----------
\subsection{Tabele pomiarowe}


\begin{table}[H]
    \centering
    \begin{tabular}{|c|c|}
        \hline
        \textbf{Nr. pomiaru} & \textbf{Wilgotność} \\
        \hline
        1 & 59\% \\
        \hline
        2 & 61\% \\
        \hline
    \end{tabular}
    \caption{Wilgotność powietrza odczytana z higrometru włosowego.}
\end{table}

\begin{table}[H]
    \centering
    \begin{tabular}{|c|c|c|c|}
        \hline
        $T_s\,[^\circ C]$ & $T_m\,[^\circ C]$ & $\Delta T\,[^\circ C]$ & wilgotność względna \\
        \hline
        25{,}00 & 20{,}00 & 5{,}00 & 63{,}00\% \\
        \hline
    \end{tabular}
    \caption{Wyniki pomiarów dla Psychrometru Assmanna.}
\end{table}

\begin{table}[H]
    \centering
    \begin{tabular}{|c|c|c|}
        \hline
        Nr. pomiaru & $T_1\,[^\circ C]$ & $T_2\,[^\circ C]$ \\
        \hline
        1 & 11{,}4 & 14{,}0 \\
        \hline
        2 & 12{,}0 & 14{,}7 \\
        \hline
        3 & 11{,}9 & 16{,}0 \\
        \hline
        4 & 12{,}3 & 13{,}8 \\
        \hline
        5 & 12{,}5 & 14{,}5 \\
        \hline
        6 & 12{,}5 & 14{,}1 \\
        \hline
        7 & 12{,}5 & 13{,}5 \\
        \hline
        8 & 11{,}9 & 14{,}4 \\
        \hline
        9 & 11{,}5 & 15{,}7 \\
        \hline
        10 & 12{,}0 & 14{,}9 \\
        \hline
        11 & 12{,}0 & 15{,}4 \\
        \hline
        12 & 12{,}3 & 14{,}9 \\
        \hline
        13 & 11{,}8 & 15{,}4 \\
        \hline
        14 & 12{,}4 & 15{,}1 \\
        \hline
        15 & 12{,}2 & 14{,}2 \\
        \hline
    \end{tabular}
    \caption{Wyniki pomiarów metodą punktu rosy.}
\end{table}


% ---------- OBLICZENIA ----------
\subsection{Temperatura punktu rosy}

Temperatura punktu rosy obliczana jest jako średnia z temperatur pojawienia i znikania mgiełki. Wartość ciśnienia pary nasyconej $p_t$ dla każdej z temperatur punktu rosy $T_{rosy}$ została odczytana ze strony internetowej~\cite{tabele} (wykorzystując równanie Antoine'a). Ciśnienie pary nasyconej dla temperatury otoczenia $T_s = 25{,}00\,^\circ C$ wynosi $p_0 = 3158\,Pa$.

Wilgotność względna $S$ obliczana jest ze wzoru:

\begin{equation}
    S= \frac{p_t}{p_0}
\end{equation}

Powyższe wartości zostały zapisane w tabeli \ref{tab:rosy}.

\begin{table}[H]
    \centering
    \begin{tabular}{|c|c|c|c|}
        \hline
        Nr. pomiaru & $T_{rosy}\,[^\circ C]$ & $p_t\,[Pa]$ & $S$ \\
        \hline
        1 & 12{,}70 & 1461{,}3 & 0{,}461 \\
        2 & 13{,}35 & 1525{,}0 & 0{,}481 \\
        3 & 13{,}95 & 1586{,}0 & 0{,}500 \\
        4 & 13{,}05 & 1495{,}3 & 0{,}472 \\
        5 & 13{,}50 & 1540{,}0 & 0{,}486 \\
        6 & 13{,}30 & 1520{,}0 & 0{,}479 \\
        7 & 13{,}00 & 1490{,}4 & 0{,}470 \\
        8 & 13{,}15 & 1505{,}0 & 0{,}475 \\
        9 & 13{,}60 & 1550{,}0 & 0{,}489 \\
        10 & 13{,}45 & 1535{,}0 & 0{,}484 \\
        11 & 13{,}70 & 1560{,}2 & 0{,}492 \\
        12 & 13{,}60 & 1550{,}0 & 0{,}489 \\
        13 & 13{,}60 & 1550{,}0 & 0{,}489 \\
        14 & 13{,}75 & 1565{,}3 & 0{,}494 \\
        15 & 13{,}20 & 1510{,}0 & 0{,}476 \\
        \hline
    \end{tabular}
    \caption{Temperatura punktu rosy oraz ciśnienie pary nasyconej dla każdego pomiaru.}
    \label{tab:rosy}
\end{table}

Średnia arytmetyczna wartości wilgotności względnej wynosi $\hat{S} = 0{,}4824$.

% ---------- NIEPEWNOŚCI ----------
\section{Ocena niepewności pomiaru}

\subsection{Złożona niepewność standardowa pomiaru wilgotności względnej}

Złożoną niepewność standardową obliczono na podstawie wzoru:

\begin{equation}
    u_c(S) = \sqrt{\frac{1}{n(n-1)} \sum_{i=1}^{n} (S_i - \hat{S})^2}
\end{equation}

gdzie:
\begin{itemize}
    \item $n = 15$ - liczba pomiarów
    \item $S_i$ - wartość wilgotności względnej dla i-tego pomiaru
    \item $\hat{S} = 0{,}4824$ - średnia wartość wilgotności względnej
\end{itemize}

Po podstawieniu wartości do wzoru otrzymujemy:
\begin{equation}
    u_c(S) = 0{,}0027
\end{equation}


% ---------- WNIOSKI ----------
\section{Wnioski}

Na podstawie przeprowadzonych pomiarów wilgotności powietrza atmosferycznego przy pomocy różnych metod pomiarowych można sformułować następujące wnioski:

\begin{enumerate}
    \item Metodą punktu rosy wyznaczono wilgotność względną powietrza:
          \begin{equation}
              S = 0{,}4824 \pm 0{,}0027
          \end{equation}
          co wyrażone w procentach daje wartość $(48{,}24 \pm 0{,}27)\%$.

    \item Za pomocą higrometru włosowego uzyskano dwa odczyty wilgotności względnej: $59\%$ na początku wykonywania pomiarów oraz $61\%$ na końcu.

    \item Psychrometr Assmanna wskazał wilgotność względną na poziomie $63\%$, przy temperaturze suchego termometru $T_s = 25{,}00^\circ C$ i mokrego termometru $T_m = 20{,}00^\circ C$.

    \item Występują znaczące różnice między wynikami uzyskanymi różnymi metodami pomiarowymi. Wartość zmierzona metodą punktu rosy (najbardziej dokładną w tym doświadczeniu) jest znacząco niższa od wartości otrzymanych pozostałymi metodami.

          % \item Mimo różnic w wartościach bezwzględnych, wszystkie metody potwierdzają, że wilgotność względna powietrza w pomieszczeniu laboratoryjnym mieściła się w zakresie od około $48\%$ do $63\%$.
\end{enumerate}

% ---------- WYKRESY ----------
\section{Wykresy}

\bibliographystyle{plain}
\bibliography{bibliography}

\end{document}
