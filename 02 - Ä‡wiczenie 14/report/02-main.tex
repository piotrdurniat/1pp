\documentclass[a4paper,12pt]{article}
\usepackage[left=2cm,right=2cm,top=2cm,bottom=2cm]{geometry} % Do ustawień marginesów
\usepackage{multicol} % Dla podziału na kolumny
\usepackage{ragged2e} % Dla justowania tekstu
\usepackage{graphicx} % Required for inserting images
\usepackage{float}
\usepackage{caption}
\usepackage{amsmath} % Math formulas
\usepackage{amssymb} % Symbols
\usepackage[svgnames]{xcolor}
\usepackage[colorlinks=true, urlcolor=blue, linkcolor=black, citecolor=orange]{hyperref} % Hyperlinks
\usepackage{polski} % Polish language
\usepackage[utf8]{inputenc} % Text encoding
\usepackage{enumitem} % Pakiet do elastycznego sterowania listami
\usepackage{indentfirst}
\usepackage{array}
\usepackage{booktabs}
\usepackage{multirow}

\begin{document}

% Górna część strony
\noindent
\begin{minipage}{0.5\textwidth}
    \raggedright
    \textbf{Piotr Durniat} \\
    I rok, Fizyka \\
    Wtorek, 8:00-10:15 \\
    \vspace{0.5cm}
    \vspace{0.5cm}
\end{minipage}%
\begin{minipage}{0.5\textwidth}
    \raggedleft
    Data wykonania pomiarów: \\
    11 marca 2025 r. \\
    \vspace{0.5cm} % Dodatkowa linia przerwy
    Prowadząca: \\
    dr Iwona Mróz 
\end{minipage}

% Tytuł ćwiczenia
\vspace{2cm} % Odstęp
\begin{center}
    \LARGE \textbf{Ćwiczenie nr 14} \\[0.5cm]
    \Large \textbf{Wyznaczanie przyspieszenia ziemskiego przy użyciu wahadła rewersyjnego}
\end{center}

% Reszta treści
\vspace{1cm} % Kolejny odstęp
\noindent

\section{Wstęp teoretyczny}
\textbf{Ruch harmoniczny} prosty to rodzaj ruchu drgającego, w którym ciało porusza się wokół położenia równowagi pod wpływem siły proporcjonalnej do wychylenia, ale przeciwnie skierowanej. Wychylenie ciała od położenia równowagi opisuje funkcja:
\[
x(t) = A \cos(\omega t + \varphi)
\]
gdzie \( A \) to amplituda drgań, \( \omega \) to częstość kołowa \( \left(\omega = 2\pi f = \frac{2\pi}{T} \right) \), \( t \) to czas, a \( \varphi \) to faza początkowa drgań.

\textbf{Wahadło matematyczne} to masa punktowa zawieszona na nierozciągliwej, nieważkiej nici o długości $l$. Dla małych kątów wychylenia, jej ruch można przybliżyć ruchem harmonicznym prostym o okresie:
\[
T = 2\pi\sqrt{\frac{l}{g}}
\]
gdzie $g$ to przyspieszenie ziemskie.

\textbf{Wahadło fizyczne} to dowolne ciało sztywne zawieszone tak, że może się wahać dookoła pewnej osi. Okres drgań wahadła fizycznego wynosi:
\[
T = 2\pi\sqrt{\frac{I}{mgl_s}}
\]
gdzie $I$ to moment bezwładności ciała względem osi obrotu, $m$ to masa ciała, a $l_s$ to odległość środka masy od osi obrotu. Wprowadzając pojęcie długości zredukowanej $l_0 = \frac{I}{ml_s}$, okres wahadła fizycznego można zapisać analogicznie do wahadła matematycznego:
\begin{equation}
    T=2\pi\sqrt{\frac{l_0}{g}}
\end{equation}

\textbf{Wahadło rewersyjne} to specjalnie skonstruowane wahadło fizyczne posiadające dwie osie obrotu, dla których okresy wahań są jednakowe. Jeśli odległość między tymi osiami wynosi $L$, to zgodnie z twierdzeniem Steinera, dla obu osi musi zachodzić warunek $l_0 = L$. 

Przyspieszenie ziemskie można wyznaczyć, mierząc okres drgań $T$ wahadła rewersyjnego i odległość $L$ między osiami obrotu, a następnie korzystając z przekształconego wzoru:
\begin{equation}
\label{eq:przyspieszenie}
g = 4\pi^2\frac{L}{T^2}
\end{equation}

Metoda ta pozwala na wyznaczenie wartości $g$ bez konieczności określania momentu bezwładności i położenia środka masy wahadła.

Podstawy teoretyczne zostały opracowane w oparciu o książkę \textit{Ćwiczenia laboratoryjne z fizyki} \cite{Drynski1976}, ze szczególnym uwzględnieniem rozdziału 17 zatytułowanego \textit{Wyznaczanie przyspieszenia ziemskiego za pomocą wahadła rewersyjnego}.

\section{Opis doświadczenia}

W ramach przeprowadzonego eksperymentu wyznaczano wartość przyspieszenia ziemskiego przy użyciu wahadła rewersyjnego oraz wahadła matematycznego. Przygotowane stanowisko składało się z wahadła rewersyjnego z dwoma ostrzami (O$_1$ i O$_2$) oraz przesuwnymi krążkami (K$_1$ i K$_2$), a także wahadła matematycznego zamontowanego na tym samym statywie.

Pomiary wykonano w następujących etapach:

\begin{enumerate}
    \item Wahadło rewersyjne zostało zawieszone na ostrzu O$_1$, tak aby krążek K$_1$ znajdował się nad punktem podparcia, a krążek K$_2$ poniżej. Następnie ustawiono krążek K$_2$ w początkowej odległości 4 cm od punktu podparcia.
    
    \item Dla każdego położenia krążka K$_2$ (zmienianego co 4 cm) zmierzono czas 20 pełnych wahnięć wahadła o małej amplitudzie i obliczono odpowiedni okres drgań $T$.
    
    \item Następnie wahadło zostało zawieszone na ostrzu O$_2$ i powtórzono pomiary czasu 20 wahnięć dla tych samych położeń krążka K$_2$.
    
    \item Na podstawie uzyskanych danych sporządzono wykresy zależności $T=f(x)$ dla obu ustawień wahadła, gdzie $x$ oznacza odległość krążka K$_2$ od punktu podparcia. Znajdując punkt przecięcia obu krzywych, wyznaczono wartości $x_0$ oraz $T_0$.
    
    \item W celu weryfikacji wyznaczono empirycznie okres drgań wahadła dla ustawienia krążka w położeniu $x_0$, zawieszając wahadło na dowolnym ostrzu.
    
    \item Dodatkowo przeprowadzono pomiar czasu 20 wahnięć wahadła matematycznego o długości $L$ (równej odległości między ostrzami O$_1$ i O$_2$) i wyznaczono jego okres drgań $T_M$.
    
    \item Dla położeń pośrednich krążka K$_2$ (29-37 cm) wykonano dokładniejsze pomiary w odstępach co 1 cm, aby precyzyjniej określić punkt przecięcia krzywych.
    
    \item W celu oszacowania niepewności pomiarowej przeprowadzono serię 10 pomiarów czasu 20 wahnięć dla stałej odległości krążka $x=20$ cm.
\end{enumerate}

Uzyskane dane posłużyły do wyznaczenia przyspieszenia ziemskiego na dwa sposoby: metodą wahadła rewersyjnego oraz metodą wahadła matematycznego, z wykorzystaniem odpowiednich wzorów (równanie \ref{eq:przyspieszenie}) oraz do analizy niepewności pomiarowych obu metod.

\section{Opracowanie wyników pomiarów}

\subsection{Tabele pomiarowe}
\begin{table}[H]
\centering
  \begin{tabular}{|c|c|c|}
  \toprule
    % \hline
        \textbf{$x$ [cm]} & \textbf{$t$ [s] dla $O_1$} & \textbf{$t$ [s] dla $O_2$} \\
        \midrule
          % \hline
        4  & 25,172 & 25,098 \\
          \hline
        8  & 24,551 & 24,037 \\
          \hline
        12 & 24,192 & 23,001 \\
          \hline
        16 & 24,022 & 21,997 \\
          \hline
        20 & 24,035 & 21,119 \\
          \hline
        24 & 24,197 & 21,463 \\
          \hline
        28 & 24,461 & 20,24  \\
          \hline
        29 & 24,537 & 20,427 \\
          \hline
        30 & 24,627 & 20,636 \\
          \hline
        31 & 24,730 & 20,931 \\
          \hline
        32 & 24,825 & 21,337 \\
          \hline
        33 & 24,932 & 21,924 \\
          \hline
        34 & 25,043 & 22,65  \\
          \hline
        35 & 25,153 & 23,65  \\
          \hline
        36 & 25,272 & 24,916 \\
          \hline
        37 & 25,402 & 26,724 \\
          \hline
        38 & 25,531 & 26,724 \\
          % \hline
        \bottomrule
    \end{tabular}
    \caption{Czas 20 wahnięć dla różnych ustawień krążka dla obu osi wahadła rewersyjnego.}
    \label{tab:czas}
\end{table}

\begin{table}[H]
    \centering
    \begin{tabular}{|c|c|c|}
        \toprule
          % \hline
        $x$ [cm] & $T$ [s] (dla osi $O_1$) & $T$ [s] (dla osi $O_2$) \\
          % \hline
        \midrule
        4  & 1,2586  & 1,2549  \\
          \hline
        8  & 1,22755 & 1,20185 \\
          \hline
        12 & 1,2096  & 1,15005 \\
          \hline
        16 & 1,2011  & 1,0997  \\
          \hline
        20 & 1,20175 & 1,05595 \\
          \hline
        24 & 1,20985 & 1,07315 \\
          \hline
        28 & 1,22305 & 1,012   \\
          \hline
        29 & 1,22685 & 1,04655 \\
          \hline
        30 & 1,23135 & 1,02135 \\
          \hline
        31 & 1,2365  & 1,0318  \\
          \hline
        32 & 1,24125 & 1,06685 \\
          \hline
        33 & 1,2466  & 1,0962  \\
          \hline
        34 & 1,25215 & 1,1325  \\
          \hline
        35 & 1,25765 & 1,1825  \\
          \hline
        36 & 1,2636  & 1,2458  \\
          \hline
        37 & 1,2701  & 1,3362  \\
          \hline
        38 & 1,27655 & 1,3362  \\
          \hline
        % \bottomrule
    \end{tabular}
    \caption{Okresy drgań wahadła rewersyjnego dla różnych ustawień krążka dla obu osi.}
    \label{tab:okresy}
\end{table}

\begin{table}[H] 
    \centering
    \begin{tabular}{|c|c|}
        \toprule
        Nr & $t(x=20\,\text{cm})$ [s] \\
        \midrule
        1.  & 21,132 \\
          \hline
        2.  & 21,140  \\
          \hline
        3. & 21,136 \\
          \hline
        4. & 21,135 \\
          \hline
        5. & 21,135 \\
          \hline
        6. & 21,136 \\
          \hline
        7. & 21,134 \\
          \hline
        8. & 21,130 \\
          \hline
        9. & 21,135 \\
          \hline
        10. & 21,133 \\
        \bottomrule
    \end{tabular}
    \caption{Czas 20 wahnięć dla odległości krążka 20 cm.}
    \label{tab:last_column}
\end{table}

\begin{table}[h]
    \centering
    \begin{tabular}{|c|c|}
        \hline
        \textbf{Typ wahadła} & \textbf{$\mathbf{t(x=37.5\text{ cm})}$ [s]} \\ 
        \hline
        Rewersyjne (Oś O1) & 25,449 \\
        \hline
        Rewersyjne (Oś O2) & 25,647 \\
        \hline
        Matematyczne & 25,766 \\ 
        \hline
    \end{tabular}
    \caption{Czas 20 wahnięć dla obu wahadeł przy $x=37,5$ cm.}
    \label{tab:czasy}
\end{table}
\newpage

\subsection{Okres drgań wahadła rewersyjnego}

Wyznaczono punkt przecięcia obu prostych. Dla odległości odpowiadającej temu punktowi ($x=37,5$ cm) wykonano pomiary czasu dla obu osi, a wyniki umieszczono w tabeli \ref{tab:czasy}. Czasy dla obu osi były bardzo podobne, lecz nieznacznie różne, więc końcowy okres drgań wahadła rewersyjnego ustalono jako średnią arytmetyczną tych dwóch czasów, podzieloną przez liczbę wahnięć. Otrzymana wartość wyniosła $T_0 = 1.280375$ s.

\subsection{Wahadło rewersyjne}

Na podstawie wykresu wyzanczony został punkt przeciecia.

- obliczenie g z wzoru

\subsection{Wahadło matematyczne}

- obliczenie średniej okresu
- obliczenie g ze wzoru


\section{Ocena niepewności pomiaru}


\subsection{Niepewność pomiaru czasu}

Do obliczenia niepewności pomiaru czasu wykonano 10 dodatkowych pomiarów czasu 20 wahnięć wahadła rewersyjnego dla stałej odległości między krążkami $x=20 \text{cm}$. Wartości znajdują się w tabeli \ref{tab:last_column}. Obliczono całkowitą niepewność standardową $u_A(t)$ na podstawie wzoru \ref{eq:u_c}, gdzie $u_A(x)$ oznacza niepewność standardową typu A obliczoną korzystając ze wzoru \ref{eq:u_A}, a $u_B(x)$ oznacza niepewność standardową typu B obliczoną ze wzoru \ref{eq:u_B}. Niepewność wzorcowania $\Delta_d t$ dla zastosowanego stopera wynosi $0.001$ s.
 
\begin{equation}
\label{eq:u_c}
u_c(x) = \sqrt{u_A^2 + u_B^2}
\end{equation}

\begin{equation}
\label{eq:u_A}
u_A(x) = \sqrt{\frac{1}{N-1} \sum_{i=1}^{N} (x_i - \bar{x})^2}
\end{equation}

\begin{equation}
\label{eq:u_B}
u_B(x) = \frac{\Delta_d x}{\sqrt{3}}
\end{equation}



Podstawiając wartości otrzymano: $u_c(t) = 0.0027\,\text{s}$.

\subsection{Niepewność pomiaru okresu}

Niepewność okresu obliczono na podstawie praw przenoszenia niepewności:

\begin{equation}
\label{eq:niepewnosc_zlozona}
    u_c(E) = \sqrt{\sum_{k=1}^{K} \left( \frac{\partial E}{\partial x_k} \right)^2 u^2(x_k)}.
\end{equation}

Okres $T$ wyrażony jest wzorem:

\[
T = \frac{t}{N}
\]

gdzie $N$ to liczba pełnych okresów. Przekształcając wzór, otrzymujemy:

\[
u_c(T) = \sqrt{\left(\frac{\partial}{\partial t} T\right)^2 u^2(t)} = \sqrt{\left(\frac{\partial}{\partial t} \frac{t}{N}\right)^2 u^2(t)} = \sqrt{ \frac{1}{N^2} \cdot u^2(t)} = \frac{u_c(t)}{N}
\]

Podstawiając wartości, otrzymujemy: $u_c(T) = 0.00027\,\text{s}$.

\subsection{Niepewność pomiaru długości wahadła}

Długość wahadła matematycznego oraz odległość między dwiema osiami wahadła rewersyjnego została zmierzona jednokrotnie za pomocą miarki o niepewności wzorcowania równej $\Delta_d x = 0.01\,\text{mm}$. Stąd, korzystając z wzoru \ref{eq:u_B} na niepewność typu B, obliczono wartość niepewności pomiaru długości wahadła, która wyniosła: $u_B(L) = 0.00058\,\text{m}$.

\subsection{Niepewność pomiaru przyspieszenia ziemskiego}

Do obliczenia niepewności przyspieszenia ziemskiego wykorzystano wzór \ref{eq:niepewnosc_przyspieszenie}:

\begin{equation}
\label{eq:niepewnosc_przyspieszenie}
u(g) = \frac{u(L)}{L} + \frac{2 u(T)}{T}
\end{equation}

Podstawiając wartości dla obu wahadeł do powyższego wzoru, otrzymano różne wartości, które przedstawiono w tabeli \ref{tab:niepewnosci}.


\begin{table}[H]
    \centering
    \begin{tabular}{|c|c|}
        \hline
        Typ wahadła & $u(g) \left[\frac{m}{s^2}\right]$ \\
        \hline
        Rewersyjne & 0.0016 \\
        \hline
        Matematyczne & 0.0012 \\
        \hline
    \end{tabular}
    \caption{Wartości niepewności przyspieszenia ziemskiego dla obu typów wahadeł.}
    \label{tab:niepewnosci}
\end{table}

\section{Wnioski}


Na podstawie przeprowadzonych pomiarów i analizy danych sformułowano następujące wnioski:

\begin{enumerate}
    \item Wyznaczona wartość przyspieszenia ziemskiego wynosi $g_{rew} = ... \pm 0,0016$ $\frac{m}{s^2}$ (metodą wahadła rewersyjnego) oraz $g_{mat} = ... \pm 0,0012$ $\frac{m}{s^2}$ (metodą wahadła matematycznego).
    
    % \item Metoda wahadła matematycznego charakteryzuje się mniejszą niepewnością pomiarową, co wskazuje na jej większą precyzję w warunkach laboratoryjnych.
    
    \item Wykres zależności $T=f(x)$ dla obu osi obrotu wahadła rewersyjnego potwierdza teoretyczne przewidywania. Punkt przecięcia krzywych przy $x_0 = 37,5$ cm odpowiada położeniu, w którym okresy drgań są równe dla obu osi.
    
    \item Różnica między zmierzonymi czasami drgań dla obu osi w punkcie przecięcia ($t_{O1} = 25,449$ s, $t_{O2} = 25,647$ s) wynika z niedokładności metody wyznaczenia tego punktu (rysowanie linii na kartce) i mogłaby zostać zminimalizowana stosując bardziej precyzyjne metody matematyczne.
    
    \item Zaletą metody wahadła rewersyjnego jest brak konieczności dokładnego określania położenia środka masy oraz momentu bezwładności wahadła, co eliminuje potencjalne źródła błędów systematycznych.
\end{enumerate}

\section{Wykresy}

Wykres został załączony na końcu sprawozdania.

\bibliographystyle{plain}
\bibliography{bibliography}

\end{document}