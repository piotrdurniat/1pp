\documentclass[a4paper,12pt]{article}
\usepackage[left=2cm,right=2cm,top=2cm,bottom=2cm]{geometry} % Do ustawień marginesów
\usepackage{multicol} % Dla podziału na kolumny
\usepackage{ragged2e} % Dla justowania tekstu
\usepackage{graphicx} % Required for inserting images
\usepackage{float}
\usepackage{caption}
\usepackage{amsmath} % Math formulas
\usepackage{amssymb} % Symbols
\usepackage[svgnames]{xcolor}
\usepackage[colorlinks=true, urlcolor=blue, linkcolor=black, citecolor=orange]{hyperref} % Hyperlinks
\usepackage{polski} % Polish language
\usepackage[utf8]{inputenc} % Text encoding
\usepackage{enumitem} % Pakiet do elastycznego sterowania listami
\usepackage{indentfirst}
\usepackage{array}

\setlist[itemize]{itemsep=0pt, topsep=0pt}

\begin{document}

% Górna część strony
\noindent
\begin{minipage}{0.5\textwidth}
    \raggedright
    \textbf{Piotr Durniat} \\
    I rok, Fizyka \\
    Wtorek, 8:00-10:15 \\
    \vspace{0.5cm}
    \vspace{0.5cm}
\end{minipage}%
\begin{minipage}{0.5\textwidth}
    \raggedleft
    Data wykonania pomiarów: \\
    06.05.2025 \\
    \vspace{0.5cm} % Dodatkowa linia przerwy
    Prowadząca: \\
    dr Iwona Mróz
\end{minipage}

% Tytuł ćwiczenia
\vspace{2cm} % Odstęp
\begin{center}
    \LARGE \textbf{Ćwiczenie nr 30} \\[0.5cm]
    \Large \textbf{Wyznaczanie względnej gęstości cieczy i ciał stałych}
\end{center}

% Reszta treści
\vspace{1cm} % Kolejny odstęp
\noindent

\tableofcontents
\newpage

% ---------- WSTĘP TEORETYCZNY ----------
\section{Wstęp teoretyczny}

\subsection*{Ciężar właściwy ciała}

Ciężar właściwy ciała ($\gamma$) jest to stosunek ciężaru ciała ($P$) do jego objętości ($V$), wyrażony wzorem:

\begin{equation*}
    \gamma = \frac{P}{V}
\end{equation*}


W ogólności ciała rozszerzają się, gdy rośnie temperatura, tym samym ponieważ masa pozostaje stała, to gęstość ciała maleje. Istnieją jednak wyjątki od tej reguły, np. woda, która w temperaturze poniżej 4 stopni Celsjusza zachowuje się anomalnie - wzrasta jej gęstość wraz ze wzrostem temperatury, a poniżej 4 stopni Celsjusza zachowuje się odwrotnie - maleje gęstość wraz ze wzrostem temperatury.

\subsection*{Waga Jolly'ego}

Wyprowadzenie wzoru na względną gęstość ciała stałego wyznaczonego za pomocą wagi Jolly'ego:

Względna gęstość ciała stałego względem cieczy:

\begin{equation}
    \label{eq:wzgledna_gestosc_jolly}
    R = \frac{\rho_{s}}{\rho_c}=  \frac{h_p - h_0}{h_p - h_w}
\end{equation}

gdzie:
\begin{itemize}
    \item $\rho_{s}$ to gęstość badanego ciała stałego,
    \item $\rho_c$ to gęstość cieczy,
    \item $h_p$ to położenie wskazówki wagi, gdy ciężarek znajduje się w powietrzu (na górnej szalce),
    \item $h_w$ to położenie wskazówki wagi, gdy ciężarek znajduje się w cieczy (na dolnej szalce),
    \item $h_0$ to położenie wskazówki wagi, gdy obie szalki są puste (bez ciężarka).
\end{itemize}


\subsection*{Moment siły}

Moment siły $M$ jest to iloczyn siły $F$ i ramienia $r$:

\begin{equation*}
    M = F \cdot r
\end{equation*}

\subsection*{Waga Mohra}

Waga Mohra składa się z wysuwanego ramienia, na którym umieszczona jest belka. Jedno z ramion belki jest podzielone na 10 równych działek, na których można wieszać ciężarki o znanych masach umownych. Na końcu belki znajduje się nurek, który umieszczany jest w badanej cieczy. Po umieszczeniu nurka w cieczy, należy zrównoważyć wagę za pomocą ciężarków umieszczonych na szalkach. Zając wagi koników i ich położenia, można wyznaczyć masę wypartej cieczy, korzystając z równowagi momentów sił.
W stanie równowagi belki zachodzi równowaga momentów sił między ciężarkiem zawieszonym na 10 podziałce, a konikiem zawieszonym na $n$-tej podziałce. Porównując momenty sił dla obu ramion belki otrzymujemy:

\begin{equation*}
    m_z gR = m_i g \frac{R}{10} n
\end{equation*}

gdzie:
\begin{itemize}
    \item $m_z$ - masa zastępcza ciężarka na 10 podziałce
    \item $m_i$ - masa $i$-tego konika
    \item $n$ - numer podziałki, na której wieszony jest konik
    \item $R$ - długość ramienia
\end{itemize}

Stąd masa zastępcza konika zawieszonego na $n$-tej podziałce wynosi:

\begin{equation*}
    m_z = m_i \frac{n}{10}
\end{equation*}

Masa wypartej cieczy przez ciężarek równa się więc sumie mas zastępczych wszystkich zawieszonych koników.

\begin{equation}
    \label{eq:waga_mohra}
    m_w = \sum_{i=1}^{N} \frac{m_i n_i}{10}
\end{equation}

gdzie $n_i$ to numer podziałki, na której zawieszony jest $i$-ty konik, a $N$ to liczba koników.


W tym doświadczeniu szukana wielkość to względna gęstość alkoholu względem gęstości wody, ponieważ objętość cieczy wypartej przez konik jest taka sama, niezależnie od cieczy, to otrzymujemy:

\begin{equation}
    \label{eq:wzgledna_gestosc}
    R = \frac{\rho_a}{\rho_w} = \frac{\frac{m_a}{V_a}}{\frac{m_w}{V_w}} = \frac{m_a}{m_w} = \frac{\sum_{i=1}^{N_a} m_{a,i} n_{a,i}}{\sum_{i=1}^{N_w} m_{w,i} n_{w,i}}
\end{equation}

gdzie $R$ to względna gęstość alkoholu względem gęstości wody, a $m_w$ oraz $m_a$ to masy wypartej cieczy odpowiednio dla wody oraz alkoholu, $m_{a, i}$ oraz $m_{w, i}$ to masy koników odpowiednio dla alkoholu oraz wody, a $n_{a, i}$ oraz $n_{w, i}$ to numery podziałek, na których zawieszony jest $i$-ty konik.

Wstęp teoretyczny opracowano na podstawie podręcznika Fizyka dla szkół wyższych, tom 2, dział Termodynamika, rozdziały 1.3 Rozszerzalność cieplna~\cite{fizyka_dla_szkół_wyższych_tom_2}, oraz rozdziały 8 i 9 podręcznika Ćwiczenia laboratoryjne z fizyki, wydanie V~\cite{Drynski1976}.

% ---------- OPIS DOŚWIADCZENIA ----------
\section{Opis doświadczenia}

Doświadczenie polega na wyznaczeniu gęstości cieczy oraz gęstości ciał stałych przy użyciu dwóch metod: wagi Mohra oraz wagi Jolly'ego.

\subsection*{Część I: Waga Mohra}
\begin{enumerate}
    \setlength{\itemsep}{0em}
    \item Zrównoważenie wagi z nurkiem w powietrzu
    \item Zanurzenie nurka w wodzie destylowanej i zrównoważenie wagi za pomocą koników o znanych masach umownych
    \item Powtórzenie pomiaru dla alkoholu
    \item Odczyt i zapisanie położenia koników dla każdej cieczy
\end{enumerate}

\subsection*{Część II: Waga Jolly'ego}
\begin{enumerate}
    \setlength{\itemsep}{0em}
    \item Przygotowanie co najmniej czterech różnych ciał stałych do badań
    \item Wyznaczenie położenia zerowego wagi ($h_0$)
    \item Ważenie ciał na górnej szalce ($h_p$)
    \item Ważenie ciał na dolnej szalce zanurzonej w wodzie ($h_w$)
    \item Powtórzenie pomiarów dla alkoholu
\end{enumerate}

\subsection*{Część III: Sprawdzenie prawa Hooke'a}
\begin{enumerate}
    \setlength{\itemsep}{0em}
    \item Ustalenie położenia zerowego wagi Jolly'ego bez zanurzania szalek w cieczy
    \item Obciążanie szalki odważnikami od 1g do 10g, z odczytem położenia wskazówki wagi przy każdym obciążeniu
    \item Powtórzenie pomiarów dla obciążeń malejących
\end{enumerate}

% ---------- OPRACOWANIE WYNIKÓW POMIARÓW ----------
\section{Opracowanie wyników pomiarów}

% ---------- TABELE ----------
\subsection{Tabele pomiarowe}

\subsubsection*{Waga Jolly'ego}

Położenie początkowe wagi Jolly'ego: $h_0 = 2{,}4$ cm.

\begin{table}[h]
    \centering
    \begin{tabular}{|c|c|c|c|}
        \hline
        Ciało & $h_0$ [cm] & $h_p$ [cm] & $h_w$ [cm] \\
        \hline
        1 & 22,3 & 25,4 & 24,2 \\
        \hline
        2 & 22,3 & 34,6 & 33,1 \\
        \hline
        3 & 22,3 & 25,4 & 25,0 \\
        \hline
        4 & 22,3 & 30,6 & 29,8 \\
        \hline
    \end{tabular}
    \caption{Pomiary dla wody}
\end{table}

\begin{table}[h]
    \centering
    \begin{tabular}{|c|c|c|c|}
        \hline
        Ciało & $h_0$ [cm] & $h_p$ [cm] & $h_w$ [cm] \\
        \hline
        1 & 22,4 & 25,4 & 24,5 \\
        \hline
        2 & 22,4 & 34,7 & 33,5 \\
        \hline
        3 & 22,4 & 25,5 & 25,1 \\
        \hline
        4 & 22,4 & 30,6 & 30,0 \\
        \hline
    \end{tabular}
    \caption{Pomiary dla alkoholu}
\end{table}

\subsubsection*{Waga Mohra}

Masy zastępcze koników oznaczono jako $m_1, m_2, m_3$, wynoszą odpowiednio:

\begin{itemize}
    \item $m_1 = 1A$
    \item $m_2 = 0{,}1 A$
    \item $m_3 = 0{,}01 A$
\end{itemize}

Pozycje i rodzaje koników dla wody destylowanej oraz alkoholu przedstawiono w poniższych tabelach.

\begin{table}[H]
    \centering
    \begin{tabular}{|c|c|}
        \hline
        Numer podziałki & Rodzaj konika \\
        \hline
        3 & $m_1$ \\
        7 & $m_1$ \\
        2 & $m_2$ \\
        4 & $m_3$ \\
        6 & $m_3$ \\
        8 & $m_3$ \\
        \hline
    \end{tabular}
    \caption{Pozycje i rodzaje koników dla wody destylowanej (waga Mohra)}
    \label{tab:waga_mohra_woda}
\end{table}

\begin{table}[H]
    \centering
    \begin{tabular}{|c|c|}
        \hline
        Numer podziałki & Rodzaj konika \\
        \hline
        1 & $m_1$ \\
        7 & $m_1$ \\
        2 & $m_2$ \\
        4 & $m_3$ \\
        5 & $m_3$ \\
        \hline
    \end{tabular}
    \caption{Pozycje i rodzaje koników dla alkoholu (waga Mohra)}
    \label{tab:waga_mohra_alkohol}
\end{table}


% ---------- OBLICZENIA ----------
\subsection{Waga Mohra}


Korzystając z wzoru \ref{eq:wzgledna_gestosc} oraz danych z tabel \ref{tab:waga_mohra_woda} oraz \ref{tab:waga_mohra_alkohol} otrzymujemy:

\begin{align*}
    R & = \frac{1 m_1 + 7 m_1 + 2 m_2 + 4 m_3 + 5 m_3}{3 m_1 + 7 m_1 + 2 m_2 + 4 m_3 + 6 m_3 + 8 m_3}                            \\
      & = \frac{8 m_1 + 2 m_2 + 9 m_3}{10 m_1 + 2 m_2 + 18 m_3}  = \frac{8 A + 0.2 A + 0.09 A}{10 A + 0.2 A + 0.18 A}  = 0{,}799
\end{align*}



% ---------- NIEPEWNOŚCI ----------
\section{Ocena niepewności pomiaru}



\subsection{Względna gęstość obliczona z wagi Mohra}

Położenia koników były zdeterminowane przez haczyki zamocowane na stałe na belce, których niepewność położenia nie jest znana (w eksperymencie nie ma możliwości ich przesunięcia, a ich położenie nie zostało zmierzone bezpośrednio). Drugim czynnikiem jest niepewność wagi koników. Koniki nie były ważone bezpośrednio (ich masy są określone w instrukcji), więc niepewność wagi koników również nie jest znana. Kolejnym czynnikiem jest dokładność odczytu położenia równowagi belki, która wynosi jedną podziałkę, lecz w doświadczeniu nie zostało zmierzone, jak wpływa to na dokładność wyznaczenia względnej gęstości.

\subsection{Względna gęstość obliczona z wagi Jolly'ego}

Niepewność wzorcowania położenia wskazówki wagi Jolly'ego wynosi:

\begin{equation*}
    \Delta_d h = 0{,}1 \text{cm}
\end{equation*}

Korzystając z prawa przenoszenia niepewności maksymalnych:

\begin{equation}
    \Delta R = \sum_{i=1}^{n} \left | \frac{\partial R}{\partial x_i} \right | \cdot \Delta x_i
\end{equation}



% ---------- WNIOSKI ----------
\section{Wnioski}

% ---------- WYKRESY ----------
\section{Wykresy}

\bibliographystyle{plain}
\bibliography{bibliography}

\end{document}
