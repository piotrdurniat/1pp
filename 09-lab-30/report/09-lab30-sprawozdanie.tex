\documentclass[a4paper,12pt]{article}
\usepackage[left=2cm,right=2cm,top=2cm,bottom=2cm]{geometry} % Do ustawień marginesów
\usepackage{multicol} % Dla podziału na kolumny
\usepackage{ragged2e} % Dla justowania tekstu
\usepackage{graphicx} % Required for inserting images
\usepackage{float}
\usepackage{caption}
\usepackage{amsmath} % Math formulas
\usepackage{amssymb} % Symbols
\usepackage[svgnames]{xcolor}
\usepackage[colorlinks=true, urlcolor=blue, linkcolor=black, citecolor=orange]{hyperref} % Hyperlinks
\usepackage{polski} % Polish language
\usepackage[utf8]{inputenc} % Text encoding
\usepackage{enumitem} % Pakiet do elastycznego sterowania listami
\usepackage{indentfirst}
\usepackage{array}

\begin{document}

% Górna część strony
\noindent
\begin{minipage}{0.5\textwidth}
    \raggedright
    \textbf{Piotr Durniat} \\
    I rok, Fizyka \\
    Wtorek, 8:00-10:15 \\
    \vspace{0.5cm}
    \vspace{0.5cm}
\end{minipage}%
\begin{minipage}{0.5\textwidth}
    \raggedleft
    Data wykonania pomiarów: \\
    06.05.2025 \\
    \vspace{0.5cm} % Dodatkowa linia przerwy
    Prowadząca: \\
    dr Iwona Mróz
\end{minipage}

% Tytuł ćwiczenia
\vspace{2cm} % Odstęp
\begin{center}
    \LARGE \textbf{Ćwiczenie nr 30} \\[0.5cm]
    \Large \textbf{Wyznaczanie względnej gęstości cieczy i ciał stałych}
\end{center}

% Reszta treści
\vspace{1cm} % Kolejny odstęp
\noindent

\tableofcontents
\newpage

% ---------- WSTĘP TEORETYCZNY ----------
\section{Wstęp teoretyczny}

Ciężar właściwy ciała jest to stosunek ciężaru ciała do jego objętości, wyrażony wzorem:

\begin{equation*}
    \gamma = \frac{P}{V}
\end{equation*}

gdzie:
\begin{itemize}
    \setlength{\itemsep}{0em}
    \item $P$ - ciężar ciała,
    \item $V$ - objętość ciała.
\end{itemize}

W ogólności ciała rozszerzają się, gdy rośnie temperatura, tym samym ponieważ masa pozostaje stała, to gęstość ciała maleje. Istnieją jednak wyjątki od tej reguły, np. woda, która w temperaturze poniżej 4 stopni Celsjusza zachowuje się anomalnie - wzrasta jej gęstość wraz ze wzrostem temperatury, a poniżej 4 stopni Celsjusza zachowuje się odwrotnie - maleje gęstość wraz ze wzrostem temperatury.

Wyprowadzenie wzoru na względną gęstość ciała stałego wyznaczonego za pomocą wagi Jolly'ego:
\ldots

Wyprowadzenie wzoru na względną gęstość ciała stałęgo wyznaczonego za pomocą wagi Mohra:
\ldots

Moment siły $M$ jest to iloczyn siły $F$ i ramienia $r$:

\begin{equation*}
    M = F \cdot r
\end{equation*}

Równowaga momentów sił\dots

Wstęp teoretyczny opracowano na podstawie podręcznika Fizyka dla szkół wyższych, tom 2, dział Termodynamika, rozdziały 1.3 Rozszerzalność ciaplna, oraz ... \cite{fizyka_dla_szkół_wyższych_tom_2}.



% ---------- OPIS DOŚWIADCZENIA ----------
\section{Opis doświadczenia}

Doświadczenie polega na wyznaczeniu gęstości cieczy oraz gęstości ciał stałych przy użyciu dwóch metod: wagi Mohra oraz wagi Jolly'ego.

\subsection*{Część I: Waga Mohra}
\begin{enumerate}
    \setlength{\itemsep}{0em}
    \item Zrównoważenie wagi z nurkiem w powietrzu
    \item Zanurzenie nurka w wodzie destylowanej i zrównoważenie wagi za pomocą koników o znanych masach umownych
    \item Powtórzenie pomiaru dla alkoholu
    \item Odczyt i zapisanie położenia koników dla każdej cieczy
\end{enumerate}

\subsection*{Część II: Waga Jolly'ego}
\begin{enumerate}
    \setlength{\itemsep}{0em}
    \item Przygotowanie co najmniej czterech różnych ciał stałych do badań
    \item Wyznaczenie położenia zerowego wagi ($h_0$)
    \item Ważenie ciał na górnej szalce ($h_p$)
    \item Ważenie ciał na dolnej szalce zanurzonej w wodzie ($h_w$)
    \item Powtórzenie pomiarów dla alkoholu
\end{enumerate}

\subsection*{Część III: Sprawdzenie prawa Hooke'a}
\begin{enumerate}
    \setlength{\itemsep}{0em}
    \item Ustalenie położenia zerowego wagi Jolly'ego bez zanurzania szalek w cieczy
    \item Obciążanie szalki odważnikami od 1g do 10g, z odczytem położenia wskazówki wagi przy każdym obciążeniu
    \item Powtórzenie pomiarów dla obciążeń malejących
\end{enumerate}

% ---------- OPRACOWANIE WYNIKÓW POMIARÓW ----------
\section{Opracowanie wyników pomiarów}

% ---------- TABELE ----------
\subsection{Tabele pomiarowe}

% ---------- OBLICZENIA ----------
\subsection{...}

% ---------- NIEPEWNOŚCI ----------
\section{Ocena niepewności pomiaru}

% ---------- WNIOSKI ----------
\section{Wnioski}

% ---------- WYKRESY ----------
\section{Wykresy}

\bibliographystyle{plain}
\bibliography{bibliography}

\end{document}
