\documentclass[a4paper,12pt]{article}
\usepackage[left=2cm,right=2cm,top=2cm,bottom=2cm]{geometry} % Do ustawień marginesów
\usepackage{multicol} % Dla podziału na kolumny
\usepackage{ragged2e} % Dla justowania tekstu
\usepackage{graphicx} % Required for inserting images
\usepackage{float}
\usepackage{caption}
\usepackage{amsmath} % Math formulas
\usepackage{amssymb} % Symbols
\usepackage[svgnames]{xcolor}
\usepackage[colorlinks=true, urlcolor=blue, linkcolor=black, citecolor=orange]{hyperref} % Hyperlinks
\usepackage{polski} % Polish language
\usepackage[utf8]{inputenc} % Text encoding
\usepackage{enumitem} % Pakiet do elastycznego sterowania listami
\usepackage{indentfirst}
\usepackage{array}

\begin{document}

% Górna część strony
\noindent
\begin{minipage}{0.5\textwidth}
    \raggedright
    \textbf{Piotr Durniat} \\
    I rok, Fizyka \\
    Wtorek, 8:00-10:15 \\
    \vspace{0.5cm}
    \vspace{0.5cm}
\end{minipage}%
\begin{minipage}{0.5\textwidth}
    \raggedleft
    Data wykonania pomiarów: \\
    29.04.2025 \\
    \vspace{0.5cm} % Dodatkowa linia przerwy
    Prowadząca: \\
    dr Iwona Mróz
\end{minipage}

% Tytuł ćwiczenia
\vspace{2cm} % Odstęp
\begin{center}
    \LARGE \textbf{Ćwiczenie nr 26} \\[0.5cm]
    \Large \textbf{Wyznaczanie ciepła właściwego ciał stałych przy użyciu kalorymetru}
\end{center}

% Reszta treści
\vspace{1cm} % Kolejny odstęp
\noindent

\tableofcontents
\newpage

% ---------- WSTĘP TEORETYCZNY ----------
\section{Wstęp teoretyczny}

Ciepło właściwe substancji $c$ określa ilość energii potrzebnej do podwyższenia temperatury jednostkowej masy ciała o jednostkę temperatury. Jest ono definiowane jako:

\begin{equation}
    c = \frac{Q}{m \cdot \Delta T}
\end{equation}

gdzie $Q$ to dostarczona energia cieplna, $m$ to masa ciała, a $\Delta T$ to zmiana temperatury.

W doświadczeniu wykorzystujemy kalorymetr, który pozwala na pomiar ciepła właściwego ciał stałych. Metoda opiera się na zasadzie bilansu cieplnego, zgodnie z którą suma ciepła oddanego i pobranego w układzie izolowanym jest równa zeru:

\begin{equation}
    Q_1 + Q_2 = 0
\end{equation}

gdzie $Q_1$ to ciepło oddane przez ciało o wyższej temperaturze (wartość ujemna), a $Q_2$ to ciepło pobrane przez ciało o niższej temperaturze (wartość dodatnia).

Dla badanego ciała stałego o masie $m_c$, temperaturze początkowej $T_c$ i cieple właściwym $c_p$, które zostaje umieszczone w wodzie o masie $m_w$, temperaturze początkowej $T_p$ i cieple właściwym $c_w$, przy uwzględnieniu pojemności cieplnej naczynka kalorymetrycznego $K_n = m_n \cdot c_n$, bilans cieplny przyjmuje postać:

\begin{equation}
    m_c \cdot c_p \cdot (T_k - T_c) + [m_w \cdot c_w + m_n \cdot c_n] \cdot (T_k - T_p) = 0
\end{equation}

gdzie $T_k$ to temperatura końcowa układu.

Przekształcając powyższe równanie, otrzymujemy wzór na ciepło właściwe badanego ciała:

\begin{equation}
    c_p = \frac{[m_w \cdot c_w + m_n \cdot c_n] \cdot (T_p - T_k)}{m_c \cdot (T_k - T_c)}
\end{equation}

Prawo Dulonga-Petita stanowi, że molowe ciepło właściwe pierwiastków stałych w temperaturze pokojowej jest w przybliżeniu stałe i wynosi około $3R \approx 25\,\frac{\text{J}}{\text{mol} \cdot \text{K}}$, gdzie $R$ to stała gazowa. Prawo to jest przybliżeniem i sprawdza się głównie dla metali i prostych substancji krystalicznych w temperaturze pokojowej.

W rzeczywistym przebiegu doświadczenia występuje wymiana ciepła z otoczeniem, co wprowadza błąd systematyczny. Aby go zminimalizować, stosuje się metodę interpolacji do wyznaczenia rzeczywistych temperatur początkowej i końcowej, analizując zmiany temperatury w czasie przed i po osiągnięciu stanu równowagi.

Wstęp teoretyczny został opracowany na podstawie podręcznika Fizyka dla szkół wyższych, Tom 2, Dział Temodynamika, rozdział 1 - Temperatura i Ciepło \cite{fizyka_dla_szkół_wyższych_tom_2}.




% ---------- OPIS DOŚWIADCZENIA ----------
\section{Opis doświadczenia}

\begin{enumerate}
    \item Zważenie badanych ciał oraz naczyńka kalorymetrycznego z mieszadełkiem.
    \item Napełnienie naczyńka wodą (do 2/3 objętości) i określenie jej masy.
    \item Ogrzanie badanego ciała w ogrzewaczu elektrycznym z termoparą do temperatury 100-105°C.
    \item Rejestracja temperatury początkowej wody w kalorymetrze przez 5 minut (pomiar co 30 sekund).
    \item Przeniesienie ogrzanego ciała do kalorymetru i pomiar zmian temperatury:
          \begin{itemize}
              \item pierwsze 5 minut: pomiar co 30 sekund
              \item następnie: pomiar co minutę
          \end{itemize}
    \item Powtórzenie procedury dla pozostałych badanych ciał.
\end{enumerate}

Doświadczenie pozwala wyznaczyć pojemność cieplną badanych ciał poprzez analizę wymiany ciepła między ogrzanym ciałem a wodą w kalorymetrze.


% ---------- OPRACOWANIE WYNIKÓW POMIARÓW ----------
\section{Opracowanie wyników pomiarów}

% ---------- TABELE ----------
\subsection{Tabele pomiarowe}

% ---------- OBLICZENIA ----------
\subsection{...}

% ---------- NIEPEWNOŚCI ----------
\section{Ocena niepewności pomiaru}

% ---------- WNIOSKI ----------
\section{Wnioski}

% ---------- WYKRESY ----------
\section{Wykresy}

\bibliographystyle{plain}
\bibliography{bibliography}

\end{document}
