\documentclass[a4paper,12pt]{article}
\usepackage[left=2cm,right=2cm,top=2cm,bottom=2cm]{geometry} % Do ustawień marginesów
\usepackage{multicol} % Dla podziału na kolumny
\usepackage{ragged2e} % Dla justowania tekstu
\usepackage{graphicx} % Required for inserting images
\usepackage{float}
\usepackage{caption}
\usepackage{amsmath} % Math formulas
\usepackage{amssymb} % Symbols
\usepackage[svgnames]{xcolor}
\usepackage[colorlinks=true, urlcolor=blue, linkcolor=black, citecolor=orange]{hyperref} % Hyperlinks
\usepackage{polski} % Polish language
\usepackage[utf8]{inputenc} % Text encoding
\usepackage{enumitem} % Pakiet do elastycznego sterowania listami
\usepackage{indentfirst}
\usepackage{array}

\begin{document}

% Górna część strony
\noindent
\begin{minipage}{0.5\textwidth}
    \raggedright
    \textbf{Piotr Durniat} \\
    I rok, Fizyka \\
    Wtorek, 8:00-10:15 \\
    \vspace{0.5cm}
    \vspace{0.5cm}
\end{minipage}%
\begin{minipage}{0.5\textwidth}
    \raggedleft
    Data wykonania pomiarów: \\
    18.03.2025 \\
    \vspace{0.5cm} % Dodatkowa linia przerwy
    Prowadząca: \\
    dr Iwona Mróz
\end{minipage}

% Tytuł ćwiczenia
\vspace{2cm} % Odstęp
\begin{center}
    \LARGE \textbf{Ćwiczenie nr XX} \\[0.5cm]
    \Large \textbf{Tytuł ćwiczenia}
\end{center}

% Reszta treści
\vspace{1cm} % Kolejny odstęp
\noindent

\tableofcontents
\newpage

% ---------- WSTĘP TEORETYCZNY ----------
\section{Wstęp teoretyczny}

% ---------- OPIS DOŚWIADCZENIA ----------
\section{Opis doświadczenia}

% ---------- OPRACOWANIE WYNIKÓW POMIARÓW ----------
\section{Opracowanie wyników pomiarów}

% ---------- TABELE ----------
\subsection{Tabele pomiarowe}




\begin{table}[H]
    \centering
    \begin{tabular}{|c|c|c|}
        \hline
        Nr & $A$ [cm] & $t(20 \text{ drgań})$ [s] \\
        \hline
        1  & 1 & 31,50 \\
        2  & 2 & 31,31 \\
        3  & 3 & 31,41 \\
        4  & 4 & 31,50 \\
        5  & 5 & 31,31 \\
        6  & 6 & 31,43 \\
        7  & 7 & 31,44 \\
        8  & 8 & 31,28 \\
        9  & 9 & 31,34 \\
        10 & 10 & 31,50 \\
        \hline
    \end{tabular}
    \caption{Zależność okresu drgań od amplitudy}
\end{table}

\begin{table}[H]
    \centering
    \begin{tabular}{|c|c|}
        \hline
        Nr & $t(20 \text{ drgań})$ [s] \\
        \hline
        1 & 31,44 \\
        2 & 31,16 \\
        3 & 31,28 \\
        4 & 31,34 \\
        5 & 31,66 \\
        \hline
    \end{tabular}
    \caption{Pomiar okresu dla $A = 5 \text{ cm}$}
\end{table}

\begin{table}[H]
    \centering
    \begin{tabular}{|c|c|c|}
        \hline
        Nr & $m$ [g] & $x$ [cm] \\
        \hline
        1  & 10  & 21,0 \\
        2  & 20  & 28,2 \\
        3  & 30  & 35,4 \\
        4  & 40  & 42,6 \\
        5  & 50  & 50,0 \\
        6  & 60  & 57,1 \\
        \hline
        7  & 60  & 57,1 \\
        8  & 50  & 50,0 \\
        9  & 40  & 42,8 \\
        10 & 30  & 35,5 \\
        11 & 20  & 28,2 \\
        12 & 10  & 21,0 \\
        \hline
    \end{tabular}
    \caption{Zależność wychylenia od masy}
\end{table}

\begin{table}[H]
    \centering
    \begin{tabular}{|c|c|c|}
        \hline
        $m$ [g] & $t(20 \text{ drgań})\,[\text{s}]$ & $t(10 \text{ drgań})\,[\text{s}]$ \\
        \hline
        10 & 25,41 & 12,81 \\
        20 & 26,13 & 13,94 \\
        30 & 29,53 & 14,81 \\
        40 & 31,47 & 15,59 \\
        50 & 33,06 & 16,60 \\
        60 & 34,72 & 17,84 \\
        $m_x$ & 33,22 & 14,91 \\
        \hline
    \end{tabular}
    \caption{Zależność okresu drgań od masy}
\end{table}



% ---------- OBLICZENIA ----------
\subsection{...}

% ---------- NIEPEWNOŚCI ----------
\section{Ocena niepewności pomiaru}

% ---------- WNIOSKI ----------
\section{Wnioski}

% ---------- WYKRESY ----------
\section{Wykresy}

\bibliographystyle{plain}
\bibliography{bibliography}

\end{document}
