\documentclass[a4paper,12pt]{article}
\usepackage[left=2cm,right=2cm,top=2cm,bottom=2cm]{geometry} % Do ustawień marginesów
\usepackage{multicol} % Dla podziału na kolumny
\usepackage{ragged2e} % Dla justowania tekstu
\usepackage{graphicx} % Required for inserting images
\usepackage{float}
\usepackage{caption}
\usepackage{amsmath} % Math formulas
\usepackage{amssymb} % Symbols
\usepackage[svgnames]{xcolor}
\usepackage[colorlinks=true, urlcolor=blue, linkcolor=black, citecolor=orange]{hyperref} % Hyperlinks
\usepackage{polski} % Polish language
\usepackage[utf8]{inputenc} % Text encoding
\usepackage{enumitem} % Pakiet do elastycznego sterowania listami
\usepackage{indentfirst}
\usepackage{array}

\setlist[itemize]{itemsep=0pt, topsep=0pt}

\begin{document}

\section{Wzory}

\subsection{Prawo przenoszenia niepewności maksymalnych}

\begin{equation}
    \Delta y = \sum_{i=1}^{n} \left | \frac{\partial f}{\partial x_i} \right | \cdot \Delta x_i
\end{equation}

\subsection{Niepewność pomiarowa współczynników prostej regresji liniowej}

Niepewności pomiarowe dla wyznaczonej prostej regresji liniowej $y = ax + b$ obliczono na podstawie odchylenia standardowego reszt $s_y$ oraz rozkładu punktów pomiarowych wzdłuż osi $x$, korzystając z następujących wzorów:

\[
    s_y = \sqrt{\frac{\sum_{i=1}^{n} (y_i - \hat{y}_i)^2}{n-2}}
\]

\[
    u_a = s_y \sqrt{\frac{n}{n \sum x_i^2 - \left( \sum x_i \right)^2}}
\]

\[
    u_b = s_y \sqrt{\frac{\sum x_i^2}{n \sum x_i^2 - \left( \sum x_i \right)^2}}
\]

gdzie $x_i$ to wartości zmiennej niezależnej, $y_i$ to wartości zmierzone, $\hat{y}_i$ to wartości przewidywane przez model regresji, a $n$ to liczba punktów pomiarowych. Dzielnik $n-2$ wynika z faktu, że model regresji liniowej ma dwa parametry ($a$ i $b$).


Obliczone wartości niepewności dla współczynników prostej regresji wynoszą:


\end{document}

