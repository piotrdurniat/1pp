\documentclass[12pt]{article}

% --- PAKIETY ---
\usepackage[utf8]{inputenc}       % polskie znaki
\usepackage[T1]{fontenc}
\usepackage[polish]{babel}        % język polski
\usepackage{amsmath}              % zaawansowana matematyka
\usepackage{amssymb}              % symbole matematyczne
\usepackage{graphicx}             % grafika
\usepackage{geometry}             % ustawienia strony
\usepackage{float}                % kontrola położenia figur
\usepackage{siunitx}              % jednostki SI
\usepackage{enumitem}             % większa kontrola nad listami

\geometry{a4paper, margin=2.5cm}

\begin{document}

\maketitle

\section*{Wstęp teoretyczny}

\subsection*{Prawo gazu doskonałego – równanie Clapeyrona}
Zachowanie gazów w warunkach zbliżonych do idealnych opisuje równanie stanu gazu doskonałego, zwane równaniem Clapeyrona:
\[
    pV = nRT,
\]
gdzie: $p$ – ciśnienie gazu, $V$ – objętość, $n$ – liczba moli, $R$ – uniwersalna stała gazowa, a $T$ – temperatura bezwzględna. Równanie to przybliża rzeczywiste właściwości powietrza atmosferycznego, traktowanego jako mieszanina gazów doskonałych, i jest punktem wyjścia do dalszej analizy pary wodnej.

\subsection*{Para nasycona wody – własności i prawo Clausiusa-Clapeyrona}
Para nasycona to para wodna będąca w równowadze termodynamicznej z cieczą – oznacza to, że szybkość parowania i kondensacji są równe. Jej ciśnienie zależy wyłącznie od temperatury. Zależność tę opisuje równanie Clausiusa-Clapeyrona:
\[
    \frac{d p}{d T} = \frac{L}{T \Delta V},
\]
gdzie $L$ to molowe ciepło parowania, a $\Delta V$ – zmiana objętości podczas przejścia fazowego. Przy założeniu, że objętość cieczy jest pomijalnie mała względem objętości pary, a para zachowuje się jak gaz doskonały, równanie to można przekształcić do postaci:
\[
    \ln p = -\frac{L}{R} \cdot \frac{1}{T} + C,
\]
co pozwala na wyznaczenie ciepła parowania na podstawie pomiarów ciśnienia pary nasyconej w funkcji temperatury.

\subsection*{Para nienasycona – przejście do stanu nasycenia}
Para nienasycona zawiera mniej cząsteczek pary wodnej niż wynosi maksymalna możliwa ilość przy danej temperaturze i ciśnieniu. Po osiągnięciu odpowiedniej ilości pary – np. przez ochłodzenie lub dodanie pary – układ osiąga stan nasycenia. Wtedy dalsze zwiększanie zawartości pary powoduje jej kondensację.

\subsection*{Wilgotność bezwzględna i względna – definicje i zależność wzajemna}
Wilgotność bezwzględna ($\rho$) to masa pary wodnej zawarta w jednostce objętości powietrza, wyrażana najczęściej w g/m³. Z kolei wilgotność względna $\varphi$ określa stopień nasycenia powietrza parą wodną i definiowana jest jako stosunek ciśnienia pary wodnej rzeczywistej $p$ do ciśnienia pary nasyconej $p_n$ w tej samej temperaturze:
\[
    \varphi = \frac{p}{p_n} \cdot 100\%.
\]
Ponieważ $p_n$ rośnie wraz z temperaturą, to przy stałej zawartości pary wilgotność względna maleje ze wzrostem temperatury. Wilgotność względna jest wielkością najczęściej wykorzystywaną w meteorologii.

\subsection*{Metody pomiaru wilgotności względnej powietrza atmosferycznego}
Wilgotność względną można mierzyć kilkoma metodami. Najczęściej stosowane to:
\begin{itemize}[label=--]
    \item \textbf{Psychrometryczna} – opiera się na pomiarze temperatury za pomocą dwóch termometrów: suchego i zwilżonego. Różnica temperatur zależy od parowania i wilgotności powietrza. Na tej podstawie, korzystając z tablic psychrometrycznych lub wzorów empirycznych, wyznacza się wilgotność względną.
    \item \textbf{Metoda punktu rosy} – polega na schładzaniu próbki powietrza aż do momentu, w którym para zaczyna się skraplać. Temperatura kondensacji to punkt rosy, z którego można wyliczyć wilgotność względną.
    \item \textbf{Higrometryczna} – stosuje się higrometry włosowe, elektroniczne lub kondensacyjne, które mierzą zmiany właściwości fizycznych materiałów zależnych od zawartości pary wodnej.
\end{itemize}

\end{document}
